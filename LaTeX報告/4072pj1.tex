\documentclass[14pt,a4paper]{report}  %紙張設定

\usepackage{xeCJK} %中文字體模組
%\setCJKmainfont{標楷體} %中文字體
\setCJKmainfont{MoeStandardKai.ttf}
%\newfontfamily\sectionef{Times New Roman}%設定英文字體
\newfontfamily\sectionef{Nimbus Roman}
\usepackage{amsmath,amssymb} %數學公式、符號
\usepackage{graphicx,subfigure} %圖形
\usepackage{type1cm} %調整字體絕對大小
\usepackage{textpos} %設定文字絕對位置
\usepackage[top=2.5truecm,bottom=2.5truecm,
left=3truecm,right=2.5truecm]{geometry}
\usepackage{titlesec} %目錄標題設定模組
\usepackage{titletoc} %目錄內容設定模組
\usepackage{textcomp} %表格設定模組
\usepackage{caption} %圖片標題設定
\usepackage{CJK} %中文模組
\usepackage{CJKnumb} %中文數字模組
%\usepackage{wallpaper} %浮水印
\usepackage{float}
\usepackage{enumerate}
\usepackage{listings} %引用程式碼
\usepackage{setspace}
\usepackage{blindtext} 
\graphicspath{{img/}} %圖片預設讀取路徑
\renewcommand{\figurename}{\large 圖} %更改圖片標題名稱


%=------------------更改標題內容----------------------=%
\titleformat{\chapter}[hang]{\center\sectionef\fontsize{20pt}{1pt}\bfseries}{}{}{}
\titleformat{\section}[hang]{\sectionef\fontsize{18pt}{2.5pt}\bfseries}{{\thesection}}{0.5em}{}[]
\titleformat{\subsection}[hang]{\sectionef\fontsize{18pt}{2.5pt}\bfseries}{{\thesubsection}}{1em}{}[]

%=------------------更改目錄內容----------------------=%
\titlecontents{chapter}[-5em]{}{\sectionef\fontsize{16pt}{3pt}\bfseries\makebox[2.5em][l]{\CJKmainfont}}{}{\titlerule*[0.7pc]{.}\contentspage}
\titlecontents{section}[3em]{}{\sectionef\fontsize{15pt}{3pt}\mdseries\makebox[1.5em][l]{\thecontentslabel}}{}{\titlerule*[0.7pc]{.}\contentspage}
\titlecontents{subsection}[5em]{}{\sectionef\fontsize{15pt}{3pt}\mdseries\makebox[2.5em][l]{{\thecontentslabel}}}{}{\titlerule*[0.7pc]{.}\contentspage}

%=------------------更改圖目錄內容----------------------=%
\titlecontents{figure}[1em]{}{\sectionef\fontsize{16pt}{3pt}\mdseries\makebox[1.5em][l]圖{\thecontentslabel}\quad}{}{\titlerule*[0.7pc]{.}\contentspage}

%=------------------章節間距----------------------=%
\titlespacing*{\chapter} {0pt}{0pt}{18pt}
\titlespacing*{\section} {0pt}{12pt}{6pt}
\titlespacing*{\subsection} {0pt}{6pt}{6pt}

%=------------------文件----------------------=%
\begin{document}
\begin{titlepage}
%\maketitle %文件資料
\begin{center}
\makebox[1.5\width][s] %設定文字盒子[方框寬度的1.5倍寬][對其方式為文字平均分分布於方框中]
{\fontsize{24pt}{2.5pt}國立虎尾科技大學}\\[15pt] %\\距離下方5pt
%{\ten{National Formosa University}}\\[20pt]
\makebox[1.2\width][s]
{\fontsize{24pt}{2.5pt}機械設計工程系暨精密機械工程科}\\[15pt]
%{\ten{Department of Mechanical Design Engineering \& Department of Junior Precision Mechanical Engineering}}\\[20pt]
\makebox[1.5\width][s]
{\fontsize{24pt}{2.5pt}專題製作報告}\\[120pt]
%{\ten{Project\quad Report}}\\[80pt]
{\fontsize{24pt}{2.5pt}網際內容管理系統}\\
%{\ten{Application of Web-based Content Management Systems}\\[5pt]
{\fontsize{24pt}{2.5pt}在精密機械工程教學與研究上的應用}\\[50pt]
%{\ten{in Teaching and Research of Precision Mechanical Engineering}}
{\fontsize{18pt}{2.5pt}\bf Application of Web-based Content Management Systems in Teaching and Research of Precision Mechanical Engineering}
\end{center}
\begin{flushleft}
\begin{LARGE}
\vspace{5em}
\hspace{32mm}\makebox[5cm][s]%空白距離32mm,設定文字盒子[寬度為5cm][對其方式為文字平均分分布於方框中]
{指導教授:\quad 嚴\quad 家\quad 銘\quad 老\quad 師}\\[6pt]
\hspace{32mm}\makebox[5cm][s]
{班\qquad 級:\quad 五\quad 精\quad 四\quad 甲}\\[6pt]
\hspace{32mm}\makebox[5cm][s]
{學\qquad 生:\quad 郭\qquad \quad 樺\quad(50733105)}\\[6pt]
\hspace{32mm}\makebox[5cm][s]
{\hspace{36.5mm}高\quad 沁\quad 安\quad(50733144)}\\[6pt]
\hspace{32mm}\makebox[5cm][s]
{\hspace{36.5mm}林\quad 冠\quad 澔\quad(50733146)}\\[6pt]
\hspace{32mm}\makebox[5cm][s]
{\hspace{36.5mm}林\quad 侑\quad 昌\quad(50733152)}\\[6pt]
\end{LARGE}
\end{flushleft}
\vspace{6em}
\fontsize{18pt}{2pt}\selectfont\centerline{\makebox[\width][s]
{中華民國\hspace{3em} 
110 \quad 年\quad 6\quad 月}}
\par
\end{titlepage}
%=------------------------誌謝----------------------=%
\addcontentsline{toc}{chapter}{誌~~~謝}
\centerline\LARGE\textbf{誌~~謝}\\
\begin{flushleft}
\fontsize{14pt}{2.5pt}\hspace{12pt}\quad 在此鄭重感謝製作以及協助本專題完成的所有人員,首先向大四學長致謝,他們不辭辛勞解決我們的提問,甚至從來沒有不耐煩,總是貼心為我們找出最佳解答。再來是我們的指導教授嚴家銘教授,他給了我們全方位的支援,提供我們解決問題的方向和建議,給予開始接觸AI的我們有個學習的方向,開會時也時不時向我們提出建議以及未來走向,同時也給了我們能自由摸索的空間及時間,最後是由本專題組員同心協力才得以完成本題目,特此感謝。
\end{flushleft}
\newpage
%=------------------------目錄----------------------=%
\renewcommand{\contentsname}{\centerline{\fontsize{18pt}{\baselineskip}\selectfont\textbf{目\quad 錄}}}
\tableofcontents  %目錄產生
\newpage
%=------------------圖表目錄產生----------------------=%
\renewcommand{\listfigurename}{\centerline{\fontsize{18pt}{\baselineskip}\selectfont\textbf{圖\quad 目\quad 錄 }}}
\newcommand{\loflabel}{圖} %定義\loflabel 文字為圖
\renewcommand{\numberline}[1]{\loflabel~#1\hspace*{0.5em}}
\listoffigures
%\newcommand{\captioname}{圖}
\newpage
\renewcommand{\listtablename}{\centerline{\fontsize{18pt}{\baselineskip}\selectfont\textbf{表\quad 目\quad 錄 }}}
\newcommand{\lotlabel}{表} %定義\lotlabel 文字為表
\renewcommand{\numberline}[1]{\lotlabel~#1\hspace*{0.5em}}
\listoftables

\end{center}

%=-------------作者簡介-----------------=%
    \addcontentsline{toc}{chapter}{作者簡介}
    \begin{center}
	\fontsize{20pt}{0em}\selectfont \bf{作者簡介}\\
	\end{center}	
	{\begin{textblock}{6}(0,0.5)
	\begin{figure}
	\includegraphics[width=1.25in]{test10} 
	\end{figure}
	\end{textblock}}
	{\renewcommand\baselinestretch{0.99}\selectfont %設定以下行距
	{\begin{textblock}{15}(3.5,0.7)%{寬度}(以左上角為原點之右移量,下移量)
	\noindent\fontsize{14pt}{0em}\selectfont \makebox[4em][s]{姓名}\enspace:\enspace
    \fontsize{14pt}{0em}\selectfont \makebox[4em][s]{李正揚}\\     \hspace*{\fill} \\
    \fontsize{14pt}{0em}\selectfont \makebox[4em][s]{學號}\enspace:\enspace
    \fontsize{14pt}{0em}\selectfont \makebox[4em][s]{40723110} \\ %\makebox為文本盒子
    \hspace*{\fill} \\
    \fontsize{14pt}{0em}\selectfont \makebox[4em][s]{畢業學校}\enspace:\enspace
    \fontsize{14pt}{0em}\selectfont \makebox[9em][s]{國立虎尾科技大學}\\
    \fontsize{14pt}{0em}\selectfont \makebox[5em][s]{\quad}\enspace\enspace
    \fontsize{14pt}{0em}\selectfont \makebox[8em][s]{機械設計工程系}\\
    \hspace*{\fill} \\
    \fontsize{14pt}{0em}\selectfont \makebox[4em][s]{經歷}\enspace:\enspace
    \end{textblock}}}
   % \hspace*{\fill} \\
   \vspace{2em}
	{\begin{textblock}{6}(0,2.3)
	\begin{figure}
	\includegraphics[width=1.15in]{test15} 
    \end{figure}
    \end{textblock}}
    {\renewcommand\baselinestretch{0.99}
    \selectfont %設定以下行距
    {\begin{textblock}{15}(3.5,2.5) %{寬度}(以左上角為原點之右移量,下移量)
\noindent\fontsize{14pt}{0em}\selectfont \makebox[4em][s]{姓名}\enspace:\enspace
\fontsize{14pt}{0em}\selectfont \makebox[4em][s]{林于哲}\\ 
\hspace*{\fill} \\
\fontsize{14pt}{0em}\selectfont \makebox[4em][s]{學號}\enspace:\enspace
\noindent\fontsize{14pt}{0em}\selectfont \makebox[4em][s]{40723115} \\ 
\hspace*{\fill} \\
\fontsize{14pt}{0em}\selectfont \makebox[4em][s]{畢業學校}\enspace:\enspace
\fontsize{14pt}{0em}\selectfont \makebox[9em][s]{國立虎尾科技大學}\\
\fontsize{14pt}{0em}\selectfont \makebox[5em][s]{\quad}\enspace\enspace
\fontsize{14pt}{0em}\selectfont \makebox[8em][s]{機械設計工程系}\\
\hspace*{\fill} \\
\fontsize{14pt}{0em}\selectfont \makebox[4em][s]{經歷}\enspace:\enspace
    \end{textblock}}}
    %\hspace*{\fill} \\
    \vspace{2em}
    {\begin{textblock}{6}(0,4.1)
    \begin{figure}
        \includegraphics[width=1.15in]{test38} %{}內是圖片文件的相對路徑
    \end{figure}
    \end{textblock}}
    {\renewcommand\baselinestretch{0.99}\selectfont %設定以下行距
    {\begin{textblock}{15}(3.5,4.3) %{寬度}(以左上角為原點之右移量,下移量)
\noindent\fontsize{14pt}{0em}\selectfont \makebox[4em][s]{姓名}\enspace:\enspace%\noindent指定首行不進行縮排
\fontsize{14pt}{0em}\selectfont \makebox[4em][s]{黃奕慶}\\ 
\hspace*{\fill} \\
\noindent\fontsize{14pt}{0em}\selectfont \makebox[4em][s]{學號}\enspace:\enspace
\noindent\fontsize{14pt}{0em}\selectfont \makebox[4em][s]{40723138} \\ %\makebox為文本盒子
\hspace*{\fill} \\
\noindent\fontsize{14pt}{0em}\selectfont \makebox[4em][s]{畢業學校}\enspace:\enspace
\noindent\fontsize{14pt}{0em}\selectfont \makebox[9em][s]{國立虎尾科技大學}\\
\noindent\fontsize{14pt}{0em}\selectfont \makebox[5em][s]{\quad}\enspace\enspace
\noindent\fontsize{14pt}{0em}\selectfont \makebox[8em][s]{機械設計工程系}\\
\hspace*{\fill} \\
\noindent\fontsize{14pt}{0em}\selectfont \makebox[4em][s]{經歷}\enspace:\enspace
    \end{textblock}}}
   % \hspace*{\fill} \\
   \vspace{2em}
    {\begin{textblock}{6}(0,5.9)
    \begin{figure}
        \includegraphics[width=1.15in]{148} %{}內是圖片文件的相對路徑
    \end{figure}
    \end{textblock}}
    {\renewcommand\baselinestretch{0.99}\selectfont %設定以下行距
    {\begin{textblock}{15}(3.5,6.1) %{寬度}(以左上角為原點之右移量,下移量)
\noindent\noindent\fontsize{14pt}{0em}\selectfont \makebox[4em][s]{姓名}\enspace:\enspace
\noindent\fontsize{14pt}{0em}\selectfont \makebox[4em][s]{鄭博鴻}\\ \hspace*{\fill} \\
\noindent\fontsize{14pt}{0em}\selectfont \makebox[4em][s]{學號}\enspace:\enspace
\noindent\fontsize{14pt}{0em}\selectfont \makebox[4em][s]{40723148} \\ \hspace*{\fill} \\
\noindent\fontsize{14pt}{0em}\selectfont \makebox[4em][s]{畢業學校}\enspace:\enspace
\noindent\fontsize{14pt}{0em}\selectfont \makebox[9em][s]{國立虎尾科技大學}\\
\noindent\fontsize{14pt}{0em}\selectfont \makebox[5em][s]{\quad}\enspace\enspace
\noindent\fontsize{14pt}{0em}\selectfont \makebox[8em][s]{機械設計工程系}\\
\hspace*{\fill} \\
\noindent\fontsize{14pt}{0em}\selectfont \makebox[4em][s]{經歷}\enspace:\enspace
    \end{textblock}}}
    %\hspace*{\fill} \\
\vspace{2em}
    {\begin{textblock}{6}(0,7.7)
    \begin{figure}
        \includegraphics[width=1.15in]{test50} %{}內是圖片文件的相對路徑
    \end{figure}
    \end{textblock}}
    \renewcommand\baselinestretch{0.99}\selectfont %設定以下行距
    {\begin{textblock}{15}(3.5,7.9) %{寬度}(以左上角為原點之右移量,下移量)
	\noindent\noindent\fontsize{14pt}{0em}\selectfont \makebox[4em][s]{姓名}\enspace:\enspace
	\noindent\fontsize{14pt}{0em}\selectfont \makebox[4em][s]{簡國龍}\\ \hspace*{\fill} \\
	\noindent\fontsize{14pt}{0em}\selectfont \makebox[4em][s]{學號}\enspace:\enspace
	\noindent\fontsize{14pt}{0em}\selectfont \makebox[4em][s]{40723150} \\ \hspace*{\fill} \\
	\noindent\fontsize{14pt}{0em}\selectfont \makebox[4em][s]{畢業學校}\enspace:\enspace
	\noindent\fontsize{14pt}{0em}\selectfont \makebox[9em][s]{國立虎尾科技大學}\\
	\noindent\fontsize{14pt}{0em}\selectfont \makebox[5em][s]{\quad}\enspace\enspace
	\noindent\fontsize{14pt}{0em}\selectfont \makebox[8em][s]{機械設計工程系}\\
	\hspace*{\fill} \\
	\noindent\fontsize{14pt}{0em}\selectfont \makebox[4em][s]{經歷}\enspace:\enspace
    \end{textblock}}
\newpage
%=----------------書背----------------------=%
\pagestyle{empty}%設定沒有頁眉和頁腳
\begin{center}
\fontsize{0.001pt}{1pt}\selectfont .\\
\vspace{4em}
\fontsize{30pt}{30pt}\selectfont 【13】 \\
\fontsize{20pt}{20pt}\selectfont
\vspace{0.5em}
分\\
類\\
編\\
號\\
\vspace{0.5em}
\hspace{-0.5em}:\\
\vspace{0.5em}
\rotatebox[origin=cc]{270}{\sectionef\LARGE \textbf{109-4-APP-3004-1}}\\ %旋轉
\vspace{0.5em}
強\\
化\\
學\\
習\\
在\\
機\\
電\\
系\\
統\\
中\\
之\\
應\\
用\\
\vspace{2em}
一\\
一\\
零\\
級\\

\end{center}
%\newpage
%\begin{landscape}  %橫式環境
%\begin{center}
%\fontsize{0.001pt}{1pt}\selectfont .
%\vspace{70mm}
%\rotatebox[origin=cc]{90}{\LARGE 【14】}\rotatebox[origin=cc]%{180}{\LARGE 1-2-APP-8765} %旋轉
%\end{center}
%\end{landscape}
\end{document}
